\documentclass[a4paper,12pt]{article}

\usepackage{abakos}
\usepackage{abntex2cite} 

%%%%%%%%%%%%%%%%%%%%%%%%%%%
%Capa da revista
%%%%%%%%%%%%%%%%%%%%%%%%%%

\setLogoRevista{figuras/logo-revista.png}
\setCheckForUpdatesUrl{http://link.com}

\newcommand{\secaoNome}{Seção do Artigo}
\newcommand{\editorNome}{Nome completo}

% Datas do artigo
\newcommand{\dataRecebido}{31/03/2025}
\newcommand{\dataAprovado}{28/05/2025}
\newcommand{\dataPublicado}{05/06/2025}

% PID e DOI
\newcommand{\pidArtigo}{10.0000/0000-0000-idartigo}
\newcommand{\doiArtigo}{https://doi.org/10.0000/0000-0000-idartigo}

% Tipo da licença:
%\tipoLicenca{CC-BY}
\tipoLicenca{CC-BY-SA}
%\tipoLicenca{CC-BY-NC}
%\tipoLicenca{CC-BY-ND}
%\tipoLicenca{CC-BY-NC-SA}
%\tipoLicenca{CC-BY-NC-ND}

% Blocos textuais
% Blocos textuais
\newcommand{\textoLicenca}{%
	Este é um artigo de acesso aberto distribuído sob os termos da licença Creative Commons Attribution (CC BY), que permite uso, distribuição e reprodução em qualquer meio, desde que o trabalho original seja devidamente citado.
}

\newcommand{\textoCopyright}{%
	© 2025 Nome da Revista. Os autores mantêm os direitos autorais e concedem à revista o direito de primeira publicação.
}

\newcommand{\textoCRediT}{%
	Conceituação: Autor A; Metodologia: Autor B; Análise de dados: Autor C; Redação – rascunho original: Autor A; Revisão e edição: Autor B e Autor C.
}

\newcommand{\textoConflito}{%
	Os autores declaram que não existem conflitos de interesse relacionados a este trabalho.
}

\newcommand{\textoFinanciamento}{%
	Esta pesquisa foi financiada pelo Conselho Nacional de Desenvolvimento Científico e Tecnológico (CNPq), processo nº 000000/2025-0.
}

\newcommand{\editorial}{\textbf{Nome da Revista}, Cidade, v. 1, n. 1, p. 00-00, Jul. 2025 - ISSN: 0000-0000 DOI: 00.00000/xxxxxxxx}  

%%%%%%%%%%%%%%%%%INFORMAÇÕES SOBRE AUTORES %%%%%%%%%%%%%%%%%%%%%%%%%%%%%%%

%%%%%%
%%Lembre-se de submeter a versão não identificada!
%%%%%%

% Define afiliações
\afiliacao{1}{Instituto X, Departamento Y, Universidade Z}
\afiliacao{2}{Laboratório ABC, Universidade DEF}

% Define autores
\autor{João da}{Silva}{1}{https://orcid.org/0000-0000-0000-0001}
\autor{Lucas}{Costa}{2}{https://orcid.org/0000-0000-0000-0002}
\autor{José}{Souza}{1}{https://orcid.org/0000-0000-0000-0003}


% E-mail de contato
\emailCorrespondencia{email\_responsavel@mail.com}


% ======= PORTUGUÊS =======
\newcommand{\monog}{Título do Artigo em Português}

\newcommand{\resumoPT}{
	Este artigo apresenta uma análise detalhada sobre a adoção de tecnologias de inteligência artificial (IA) em ambientes educacionais, com foco na personalização do ensino e no aprimoramento do processo de aprendizagem. O objeto de estudo abrange escolas públicas e privadas de ensino médio que implementaram sistemas baseados em IA nos últimos cinco anos. O principal objetivo da pesquisa é compreender de que forma essas tecnologias têm impactado o desempenho dos estudantes, a atuação dos professores e a gestão pedagógica das instituições. A metodologia utilizada é de abordagem mista, combinando análise quantitativa de dados de desempenho escolar com entrevistas qualitativas realizadas com professores, alunos e gestores. Os resultados revelam que a personalização do conteúdo e o feedback imediato proporcionado por ferramentas baseadas em IA contribuíram significativamente para o aumento da motivação dos estudantes e para a redução das taxas de evasão. Além disso, os professores relataram maior eficiência no planejamento das aulas e no acompanhamento do progresso individual dos alunos. Contudo, também foram identificados desafios importantes, como a necessidade de formação continuada dos docentes e a garantia de equidade no acesso às tecnologias. As considerações finais apontam para a importância de políticas públicas que incentivem a adoção ética e estratégica da IA na educação, assegurando que seu uso seja voltado para a inclusão e o desenvolvimento humano. O estudo contribui com subsídios para futuras pesquisas e para a formulação de práticas educacionais inovadoras e sustentáveis.
}

\newcommand{\keywordsPT}{termo1; termo2; termo3; termo4; termo5.}

% ======= INGLÊS =======
\newcommand{\monogIngles}{Título do artigo em Inglês}
%
\newcommand{\resumoEN}{
	This article presents a detailed analysis of the adoption of artificial intelligence (AI) technologies in educational environments, focusing on the personalization of teaching and the improvement of the learning process. The study covers public and private high schools that have implemented AI-based systems over the past five years. The main objective of the research is to understand how these technologies have impacted student performance, teacher work, and the pedagogical management of institutions. The methodology follows a mixed approach, combining quantitative analysis of academic performance data with qualitative interviews conducted with teachers, students, and administrators. The results show that content personalization and the immediate feedback provided by AI-based tools have significantly contributed to increased student motivation and reduced dropout rates. Furthermore, teachers reported greater efficiency in lesson planning and in monitoring individual student progress. However, important challenges were also identified, such as the need for ongoing teacher training and ensuring equity in access to technology. The final considerations highlight the importance of public policies that encourage the ethical and strategic adoption of AI in education, ensuring its use is aimed at inclusion and human development. The study offers valuable insights for future research and for the design of innovative and sustainable educational practices.
}

\newcommand{\keywordsEN}{term1; term2; term3; term4; term5.}

% ======= ESPANHOL =======
\newcommand{\monogEspanhol}{Título do artigo em Espanhol}

\newcommand{\resumoES}{
	Este artículo presenta un análisis detallado sobre la adopción de tecnologías de inteligencia artificial (IA) en entornos educativos, con un enfoque en la personalización de la enseñanza y en la mejora del proceso de aprendizaje. El objeto de estudio abarca escuelas públicas y privadas de educación secundaria que han implementado sistemas basados en IA en los últimos cinco años. El principal objetivo de la investigación es comprender de qué manera estas tecnologías han impactado en el rendimiento de los estudiantes, en la labor de los docentes y en la gestión pedagógica de las instituciones. La metodología utilizada es de enfoque mixto, combinando el análisis cuantitativo de datos de rendimiento escolar con entrevistas cualitativas realizadas a docentes, estudiantes y directivos. Los resultados revelan que la personalización de los contenidos y la retroalimentación inmediata proporcionada por herramientas basadas en IA han contribuido significativamente al aumento de la motivación estudiantil y a la reducción de las tasas de abandono escolar. Además, los docentes informaron una mayor eficiencia en la planificación de las clases y en el seguimiento del progreso individual de los alumnos. No obstante, también se identificaron desafíos importantes, como la necesidad de formación continua para el profesorado y la garantía de equidad en el acceso a las tecnologías. Las consideraciones finales señalan la importancia de políticas públicas que fomenten la adopción ética y estratégica de la IA en la educación, asegurando que su uso se oriente hacia la inclusión y el desarrollo humano. El estudio aporta elementos para futuras investigaciones y para la formulación de prácticas educativas innovadoras y sostenibles.
}

\newcommand{\keywordsES}{término1; término2; término3; término4; término5.}

% Como citar
\newcommand{\comoCitar}{%
	\textit{Nome da Revista}, Cidade, v. 23, p. x--xx, 2025. DOI: https://doi.org/10.0000/0000-0000-0000.
}

\begin{document}
	
	% %%%%%%%%%%%%%%%%%%%%%%%%%%%%%%%%%%
	% %% Pagina de titulo
	% %%%%%%%%%%%%%%%%%%%%%%%%%%%%%%%%%%
	
	\revistaheader
	\newpage
	
	\selectlanguage{brazilian}
	\onehalfspace  % espaçamento 1.5 entre linhas
	\setlength{\parindent}{1.25cm}
	
	%%%%%%%%%%%%%%%%%%%%%%%%%%%%%%%%%%%%%%%%%%%%%%%%%
	%% INICIO DO TEXTO
	%%%%%%%%%%%%%%%%%%%%%%%%%%%%%%%%%%%%%%%%%%%%%%%%%
	
	\section{Introdução}
	
	Este documento apresenta o modelo dos manuscritos a serem submetidos e possivelmente publicados na revista Abakós. Este template\footnote{Todas as palavras em língua estrangeira devem vir em itálico, com exceção do abstract e títulos de obras na lista de referência} segue o formato desejado. Portanto, os autores podem editar este documento incluindo o conteúdo do seu trabalho científico, tomando o cuidado de preservar a formatação.
		
	\begin{citacao}
		A citação longa deve ser formatada com recuo à esquerda de 4 cm,
		fonte menor que o corpo do texto e espaçamento simples, conforme as
		normas da ABNT. Este exemplo demonstra o comportamento do ambiente,
		mostrando como o texto fica visualmente destacado em relação ao
		restante do conteúdo, permitindo a identificação imediata de uma
		citação direta longa no documento.
	\end{citacao}
	
	De forma geral, todo o texto de desenvolvimento do artigo, da introdução às considerações finais, deve estar de acordo com as seguintes diretrizes:
	\begin{itemize}
		\item Parágrafos recuados a 1,25 centímetros;
		\item Texto com tamanho 12;
		\item Fonte Times New Roman;
		\item Espaçamento entre linhas de 1,5;
		\item Texto em formato justificado.
	\end{itemize}
	
	No restante deste documento, a Seção~\ref{sec:sec} apresenta as diretrizes para a formatação de seções e subseções. A Seção~\ref{sec:figTabEqAlg}, por sua vez, apresenta as diretrizes para a inclusão de figuras, tabelas, equações e algoritmos como parte do texto do artigo. Em seguida, a Seção~\ref{sec:ref} apresenta as diretrizes sobre o formato da lista de referências e da citação de artigos como parte do texto. Por fim, a Seção~\ref{sec:con} apresenta diretrizes sobre a seção de conclusão do trabalho.
	
	
	\section{Formatação de Seções e Subseções}
	\label{sec:sec}
	
	Em todo o texto, os títulos das seções devem utilizar a formatação caixa alta, negrito, tamanho 12. Todo título de seção ou subseção deverá ser seguido de texto. Para as seções textuais utilizar numeração progressiva em algarismos arábicos, limitada até a seção quinária. Devem ser diferenciadas utilizando os recursos gráficos \cite{manualpucartigo}.
	
	Teste 1:~\cite{ponciano2018agreement, ferreira}. 
	
	Teste 2:~\citeonline{ponciano2018agreement}. 
	
	
	Os títulos das subseções secundárias são formatados em caixa baixa, negrito, tamanho 12. Os títulos das subseções terciárias são formatados em caixa baixa, itálico, negrito, tamanho 12. Por fim, os títulos das subseções quaternária são formatados em caixa baixa, sublinhado, negrito, tamanho 12. Como exemplo, seguem as subseções apresentadas nesta seção com títulos no formato desejado.
	
	\subsection{Exemplo de Subseção Secundária}
	\subsubsection{Exemplo de Subseção Terciária}
	\subsubsubsection{Exemplo de Subseção Quaternária}
	
	
	\section{Figuras, Tabelas, Equações e Algoritmos}
	\label{sec:figTabEqAlg}
	
	Imagens, tabelas, algoritmos devem ser incluídos de forma centralizada, dentro das margens. Para gráficos, quadros e tabelas, cujos dados foram extraídos da própria pesquisa, deve-se  usar a expressão “Dados da pesquisa” ao indicar a fonte.
	
	\subsection{Figuras}
	
	Telas de \textit{software}, mapas, gráficos e diagramas devem ser inseridas como figuras, indicando a fonte e referenciadas no texto. A Figura \ref{fig:figura1} apresenta o processo de revisão de trabalhos submetidos à revista Abakós, como um exemplo de figura que contém diagrama. Observe que, no caso dessa figura, a fonte é indicada como ``Dados da Pesquisa'', indicando que a figura não foi obtida de terceiros, mas elaborada pelos próprios autores, no caso, os editores da revista.
	
	% Figura
	\begin{figure}[htb]
		\centering	
		\caption[Processo de Revisão da Revista Abakós.]{Processo de revisão da revista Abakós}
		\label{fig:figura1}
		\includegraphics[width=0.8\textwidth]{figuras/Abakos-review.png}\\
		\textbf{\footnotesize Fonte: Dados da Pesquisa.}
	\end{figure}
	
	A Figura~\ref{fig:figura2}, por sua vez, é um mapa que destaca a localização do Instituto de Ciências Exatas (ICEI) no campus da PUC Minas no Coração Eucarístico, na cidade de Belo Horizonte, estado de Minas Gerais, Brasil. Observe que, no caso dessa figura, a fonte é indicada como “{\it Google Maps}	\footnote{Disponível em: \url{https://www.google.com.br/maps}.Acessado em: 12 de abr. de 2021.}”, que é um sistema de pesquisa e visualização de mapas e imagens de satélite. Encontram-se indicados também o endereço eletrônico de tal sistema e a data em que a captura foi feita. Sempre atentar para os direitos autorais das figuras de terceiros utilizadas no artigo, isto é responsabilidade dos autores.
	
	\begin{figure}[ht]
		\centering	
		\caption[Mapa com a localização do ICEI.]{Mapa com a localização do ICEI no campus da PUC Minas}
		\label{fig:figura2}
		\includegraphics[width=0.6\textwidth]{figuras/Mapa-ICEI-PUCMinas.png}\\
		\textbf{\footnotesize Fonte: Google Maps.}
	\end{figure}
	
	A Figura~\ref{fig:figura3} é um exemplo de como inserir um gráfico no corpo do texto. É importante observar que o gráfico segue o mesmo padrão das outras figuras. Cada figura inserida precisa ser citada e contextualizada no texto. Por exemplo, a Figura~\ref{fig:figura3} mostra que, para o período entre os anos de 2014 a 2020, o número médio anual de autores por artigo publicado na revista Abakós foi de aproximadamente três autores por artigo, com maior oscilação nos anos de 2014 a 2016.
	
	\begin{figure}[ht]
		\centering	
		\caption[Média de Autores por Artigo.]{Número médio de autores em artigos publicados na revista Abakós por ano, do ano de 2014 ao ano de 2020}
		\label{fig:figura3}
		\includegraphics[width=0.8\textwidth]{figuras/AutoresPorArtigoNaRevistaAbakos.png}\\
		\textbf{\footnotesize Fonte: Dados da Pesquisa.}
	\end{figure}
	
	
	
	\subsection{Tabelas}
	
	Informações tabuladas em linhas e colunas devem ser apresentadas no texto como tabelas, de modo a facilitar a análise dos dados. Por exemplo, tabelas que apresentam dados estatísticos devem ser abertas nas laterais, com espaços verticais separando as colunas e sem espaços horizontais, exceto na separação do cabeçalho. Dados textuais esquemáticos, comparativos ou descritivos que estejam apresentados em formato de linhas e colunas também devem ser incluídos no texto como tabelas. A Tabela \ref{tab:tabela1} apresenta a quantidade de artigos e edições publicadas pela revista Abakós por ano entre 2016 e 2020, como um exemplo de tabela.
	
	% Tabela
	\begin{table}[ht]
		\centering
		\caption{Número de artigos e edições publicados pela revista Abakós entre 2016 e 2020}
		\label{tab:tabela1}
		% Conteúdo da tabela
		\begin{tabular}{l|c|c}
			\hline
			\textbf{Ano}	& \textbf{Número de artigos} & \textbf{Número de edições} \\
			\hline
			2020	& 11 &  2 \\
			2019	& 12 &  2 \\
			2018	& 10 &  2 \\
			2017   & 10  &  2 \\
			2016	& 10  &  2 \\
			\hline
		\end{tabular}
		{\footnotesize\\ \textbf{Fonte: Dados da Pesquisa.}}
	\end{table}
	
	
	\subsection{\esp Equações}
	
	As equações podem ser apresentadas dentro do texto, usando o formato matemático. Esse é o caso de descrever $y=x^2+1$, pois é parte da frase e está dentro do parágrafo. Idealmente, equações mais complexas devem ser apresentadas fora do fluxo do texto e referenciadas no texto pelos seus números. Esse é o caso da Equação~\ref{eq:aid}, da Equação~\ref{eq:bid} e da Equação~\ref{eq:qscore}, que foram propostas na literatura como formas de calcular a credibilidade de seres humanos ao realizarem computação em seus sistemas cognitivos~\cite{ponciano2018agreement}. 
	
	\begin{equation}
		\label{eq:aid}
		y_{i,d}=\frac{\displaystyle\sum_{w\in Y_{i,d}}s_{w,d}}{|Y_{i,d}|}
	\end{equation}
	
	\begin{equation}
		\label{eq:bid}
		x_{i,d}=\frac{\displaystyle\sum_{w\in X_{i,d}}s_{w,d}}{|X_{i,d}|}
	\end{equation}
	
	\begin{equation}
		\label{eq:qscore}
		r_{i,d} = \frac{s_{i,d} + y_{i,d}-x_{i,d} + 1}{3}, r_{i,d}\in[0,1]
	\end{equation}
	
	\subsection{Algoritmos}
	
	Algoritmos podem ser inseridos no texto texto e referenciados pelo seu número. Esse é o caso do Algoritmo~\ref{alg:max}. O algoritmo deve explicitar as entradas (\textit{inputs}) e saídas (\textit{outputs}). No caso do Algoritmo~\ref{alg:max}, a entrada é um vetor de números ($list$) e a saída é o maior valor no vetor (definido como $biggest\_item$). Cada linha dos passos de computação do algoritmo deve ser numerada. O pacote \LaTeX~ recomendado é o \textit{algorithmic}, que é usado nesse exemplo.
	
	\begin{algorithm}
		\caption{Valor Máximo no Vetor}
		\label{alg:max}
		\begin{algorithmic}[1]
			\REQUIRE $list$
			\ENSURE $biggest\_item$
			\STATE $biggest\_item \Leftarrow list[0]$
			\FORALL{$item$ in $list$}
			\IF{$item > biggest\_item$}
			\STATE $biggest\_item \Leftarrow item$
			\ENDIF
			\ENDFOR
		\end{algorithmic}
	\end{algorithm}
	
	\section{Dicas: Citações e Referências}
	\label{sec:ref}
	
	Tanto as citações como as referências citadas no texto devem seguir as normas da ABNT, recomenda-se fortemente consultar o documento de Elaboração de Referências~\cite{manualpucref} \footnote{Disponível em: \url{http://portal.pucminas.br/biblioteca/documentos/Guia-ABNT-referencias.pdf}. Acesso em: 06 de Mai. de 2021.} e ao documento de citações e referências~\cite{manualpuccit}\footnote{Disponível em: \url{http://portal.pucminas.br/biblioteca/documentos/citacoes-referencias.pdf}. Acesso em: 06 de Mai. de 2021.}.
	
	\subsection{Como fazer citações no texto}
	
	Para citar trabalhos científicos (artigo, livro, etc) no texto, deve-se trazer o sobrenome do autor/autores e a data da publicação, como nos exemplos abaixo:
	
	\begin{itemize}
		\item ``De acordo com Dickman e Ponciano (2020)...'' \textit{[Em letras minúsculas caso venha no texto]}
		\item ``Os dados divergiram como previsto em trabalhos da literatura (DICKMAN; PONCIANO, 2020)'' \textit{[Em caixa alta caso venha em parênteses]}
		\item Para três autores ou mais deve-se usar ``Dickman et al. (2020)'', no texto ou "(DICKMAN et al., 2020)".
	\end{itemize}
	
	Todas as referências citadas no texto devem ser listadas no final do trabalho. Observe que trabalhos que não foram citados NÃO devem ser relacionados na lista final.
	
	\subsection{Como elaborar a lista de referências}
	
	As referências devem ser organizadas em ordem alfabética de acordo com o último sobrenome do primeiro autor. Caso haja mais de uma obra do mesmo autor, a ordem deve obedecer a ordem de chamada no texto.
	
	É importante observar que uniformizamos as citações de modo que apenas as iniciais dos nomes devem ser utilizadas, como em: DICKMAN, A. G.; PONCIANO, L. 
	
	A seguir mostramos como montar a lista de referências, para maiores detalhes consultar os {\it links} disponibilizados acima:
	
	\begin{itemize}
		\item Para livro~\cite{arroyo}: \\
		
		\noindent ARROYO, M. G. {\bf Ofício de mestre}: Imagens e auto-imagens. 15.ed. Petrópolis: Vozes, 2013.
		
		\item Para capítulo de livro~\cite{bordieu}: \\
		
		\noindent BORDIEU, P. Espaço social e espaço simbólico. {\it In}: BORDIEU, P. {\bf Razões práticas}: sobre a teoria da ação. 11.ed. Tradução: Mariza Corrêa. Campinas: Papirus, 2011. cap.1, pp.13.27.
		
		\item Para artigo em periódico~\cite{ferreira}: \\
		
		\noindent FERREIRA, A.C.; DICKMAN, A.G. História oral: um método para investigar o ensino de física para estudantes cegos. {\bf Revista Brasileira de Educação Especial}, v.21, n.2, p.245-58, 2015.
		
		\item Para trabalho em atas/anais de eventos~\cite{kogut}:
		
		\noindent KOGUT, M.C. A formação docente: Os saberes e a identidade do professor. {\it In}: CONGRESSO NACIONAL DE EDUCAÇÃO, 12., 2015, Curitiba. Anais {\bf […]}. Curitiba: PUCPR, 2015.
		
		\item Para dissertação~\cite{exemplodiss}:
		\noindent PERTENCE, M.L.B. {\bf Ensino de física para estudantes cegos}: Oficina sobre criação e adaptação de materiais didáticos para professores e licenciandos. 2021. Dissertação (Mestrado em Ensino) - Pontifícia Universidade Católica de Minas Gerais, Belo Horizonte, 2021.
		
		\item Para tese~\cite{exemplotese}:
		\noindent DICKMAN, A.G. {\bf Transições de Fases de Não-equilíbrio em Sistemas de Partículas Interagentes} 1996. Tese (Doutorado em Física) - Universidade Federal de Minas Gerais, Belo Horizonte, 1996.
	\end{itemize}
	
	Na lista de referências, estão incluídos exemplos de artigos em anais de eventos~\cite{ponciano2017designing}, artigos em periódicos~\cite{ponciano2018agreement}, livro~\cite{swokowski1994} e capítulo de livros.
	
	\section{Considerações Finais}
	\label{sec:con}
	
	A seção de Considerações Finais (ou Conclusão) deve estar de acordo com os objetivos do trabalho. Ela não deve apresentar novos resultados, citações ou interpretações de outros autores. Uma conclusão pode ser estruturada em partes. A primeira parte é um sumário do trabalho, relembrando o problema, objetivos e abordagem metodológica empregada. A segunda parte é uma descrição dos principais resultados e contribuições que se encontram reportados no trabalho. É importante relacionar os resultados obtidos com aqueles encontrados na literatura, discutidos no referencial teórico do trabalho. Por fim, a terceira parte é uma discussão de trabalhos relacionados que se mostram promissores a partir dos resultados obtidos.
	
	
	%%%%%%%%%%%%%%%%%%%%%%%%%%%%%%%%%%%
	%% FIM DO TEXTO
	%%%%%%%%%%%%%%%%%%%%%%%%%%%%%%%%%%%
	
	%%%%%%%%%%%%%%%%%%%%%%%%%%%%%%%%%%%
	%% Inicio bibliografia
	%%%%%%%%%%%%%%%%%%%%%%%%%%%%%%%%%%%
	
	\newpage
	
	\bibliographystyle{abntex2-alf}
	\bibliography{bibliografia}
	
\end{document}